\documentclass[a4paper]{extarticle}
\usepackage[left=3cm,right=1cm,
    top=2cm,bottom=2cm,bindingoffset=0cm]{geometry}
\usepackage[T2A]{fontenc}
\usepackage[utf8]{inputenc}
\usepackage[english,russian]{babel}
\usepackage{indentfirst}
\usepackage[final]{graphicx}
\usepackage{algorithm}
\usepackage{algorithmic}
\usepackage{multirow}
\usepackage{longtable}
\usepackage{tabulary}
\usepackage{placeins}
\usepackage{needspace}
\usepackage[noae]{Sweave}

\author{Пахтусов Н. Г., ПРО-306}
\makeatletter



\begin{document} % Тут начинается отчет

\begin{center}
\thispagestyle{empty} 

Федеральное государственное бюджетное образовательное учереждение высшего \\
профессионального образования\\
<<Уфимский государственный авиационный технический университет>>
\vspace*{\fill}
\begingroup
\centering

Отчет по лабораторной работе №5\\
по дисциплине: <<Статистическое моделирование>>\\
на тему: <<Временные ряды>>.

\endgroup
\vspace*{\fill}

\end{center}

\begin{flushright}

Выполнил:\\
\@author \\
Проверила: \\
Рассадникова Е. Ю.

\end{flushright}

\begin{center}
Уфа, 2015 г.
\end{center}

\clearpage
\tableofcontents
\clearpage

\input{test-concordance}

\section{Постановка задачи}

В данной работе нам необходимо для некоторых временных рядов построить модель ARIMA и провести её улучшение (если это необходимо).

\section{Работа с временными рядами в R}


В базовые возможности R входят средства для представления и анализа временных рядов. Основным типом временных данных является
<<\texttt{ts}>>, который представляет собой временной ряд, состоящий из значе-
ний, разделенных одинаковыми интервалами времени. Временные ряды могут быть образованы и неравномерно отстоящими друг от друга значения. В этом случае следует воспользоваться специальными типами данных — zoo и its, которые становятся доступными после загрузки
пакетов с теми же именами.

Методы для анализа временных рядов и их моделирования включают ARIMA-модели, реализованные в функциях \texttt{arima()}, \texttt{AR()} и \texttt{VAR()}, структурные модели в \texttt{StructTS()}, функции автокорреляции и частной автокорреляции в \texttt{acf()} и \texttt{pacf()}, классическую декомпозицию временного ряда в \texttt{decompose()}, STL-декомпозицию в \texttt{stl()}, скользящее среднее и авторегрессивный фильтр в \texttt{filter()}.

\section{Данные}

В качестве исходных данных будем использовать данные о смертности в США за каждый месяц, начиная с января 1973 года и заканчивая декабрём 1978. Данные взяты с сайта https://datamarket.com/

Загрузим исходные данные:

\begin{Schunk}
\begin{Sinput}
> fram <- read.csv("/home/nick/Projects/R/Lab5/data.csv")
\end{Sinput}
\end{Schunk}

Запишем данные в переменную \texttt{data} данные из столбца с названием \texttt{data}, укажем, что начинаем с 1 месяца 1973 года, и что всего у нас 12 месяцев:
\begin{Schunk}
\begin{Sinput}
> data <- ts(fram$data,
+    start = c(1973, 1),
+    frequency = 12
+ )
\end{Sinput}
\begin{Soutput}
       Jan   Feb   Mar   Apr   May   Jun   Jul   Aug   Sep   Oct   Nov   Dec
1973  9007  8106  8928  9137 10017 10826 11317 10744  9713  9938  9161  8927
1974  7750  6981  8038  8422  8714  9512 10120  9823  8743  9129  8710  8680
1975  8162  7306  8124  7870  9387  9556 10093  9620  8285  8433  8160  8034
1976  7717  7461  7776  7925  8634  8945 10078  9179  8037  8488  7874  8647
1977  7792  6957  7726  8106  8890  9299 10625  9302  8314  8850  8265  8796
1978  7836  6892  7791  8129  9115  9434 10484  9827  9110  9070  8633  9240
\end{Soutput}
\end{Schunk}

%Построим график из получившихся данных:

%<<fig=TRUE>>=
%plot(data)
%@

\section{Начальный анализ}

Теперь можно приступить к анализу. 
\subsection{Автокорреляция и периодичность}
Применим функцию \texttt{acf} (<<auto-correlation function>>, ACF). Она выводит коэффициенты автокорреляции и рисует график автокорреляции. Чем больше палочек выходит за полоски, тем более значима периодичность.

\begin{Schunk}
\begin{Sinput}
> acf(data, main="")
\end{Sinput}
\end{Schunk}
\includegraphics{test-004}

\subsection{Тренд}

То, что волнообразный график пиков как бы затухает, говорит о том, что в наших данных возможен тренд. Проверим это с помощью функции \texttt{stl} (STL -- <<Seasonal Decomposition of Time Series by Loess>>), которая вычленяет из временного ряда три компоненты: сезонную (в данном случае, годовую), тренд и случайную, при помощи сглаживания данных методом LOESS.

\begin{Schunk}
\begin{Sinput}
> plot(stl(data, s.window="periodic")$time.series, main="")
\end{Sinput}
\end{Schunk}
\includegraphics{test-005}

Из графика видно, что тенденция к уменьшению смертности практически не прослеживается.

\subsection{Стационарность}

Проверим ряд на стационарность, использовав тест Дики -- Фуллера (DF-test). В этом нам поможет функция из пакета <<\texttt{tseries}>> \texttt{adf.test}. На вход ей подаётся временной ряд, гипотеза, которую нужно проверить и периодичность (принято брать годовую, а у нас месячные данные, значит, k = 12).

\begin{Schunk}
\begin{Sinput}
> adf.test(data, alternative="stationary", k=12)
\end{Sinput}
\begin{Soutput}
	Augmented Dickey-Fuller Test

data:  data
Dickey-Fuller = -1.6502, Lag order = 12, p-value = 0.7178
alternative hypothesis: stationary
\end{Soutput}
\end{Schunk}

Значение p оказалось равным 0.7178. Таким образом, наш временной ряд стационарен с высокой вероятностью и имеет степень интегрированности \texttt{I(0)}.
%Если p = 1, то процесс имеет единичный корень, в этом случае ряд y(k) не стационарен, а степень интегрированности процесса равна 1. Степень интегрированности обозначается как I(1).

%Если 0 < p < 1, то ряд стационарный и имеет степень интегрированости I(0)

%Для финансово-экономических процессов значение p > 1 не свойственно, так как в этом случае процесс является "взрывным". Возникновение таких процессов маловероятно, так как финансово-экономическая среда достаточно инерционная, что не позволяет принимать бесконечно большие значения за малые промежутки времени.
         
\section{Построение ARIMA модели}

Построим модель временного ряда общего числа смертей распространенным методом ARIMA (<<Autoregressive Integrated Moving Average>>, авторегрессия интегрированного скользящего среднего). Аргумент \texttt{order} отвечает за несезонную часть модели ARIMA. Он состоит из трёх значений -- (p, d, q), где p является порядком авторегрессионной модели (AR), d отвечает за степень интегрирования, а q -- за порядок модели скользящего среднего (MA). Таким образом, необходмо выбрать оптимальное значение параметра \texttt{order}. Для этого мы будем перебирать различные значения каждого из трех
его компонентов. В этом нам поможет функция \texttt{AIC}. AIC (Akaike’s Information Criterion) -- информационный критерий Акаике. Это критерий, который позволяет сравнивать модели между собой. Чем меньше AIC, тем лучше модель.

Для начала, проверим, соответствуют ли остатки модели <<белому шуму>> на модели с параметром \texttt{order} = (0,0,0) с помощью функции \texttt{tsdiag}:

\begin{Schunk}
\begin{Sinput}
> tsdiag(arima(data, order = c(0,0,0)), 100)
\end{Sinput}
\end{Schunk}
\includegraphics{test-007}

Из графиков выше можно заметить, что остатки зависимы (это видно по ACF). p-value ниже уровня значимости, значит, неизвестно, являются ли остатки белым шумом.

\subsection{Нахождения оптимальной модели}

Найдём оптимальную модель:

\begin{enumerate}
\item Коэффициент MA:
\begin{Schunk}
\begin{Sinput}
> mini = 100500
> minInd = 0
> first <- 0
> for (m in 10:16)#должно быть 1:length(data)
+ {
+      mm <- arima(data, order=c(0,0,m))
+      cur = AIC(mm)
+      if(cur < mini){
+        minInd = m
+        mini = cur
+        first <- minInd
+      }
+ }
\end{Sinput}
\end{Schunk}

MA-коэффицент получился равным 15:
\begin{Schunk}
\begin{Sinput}
> first
\end{Sinput}
\begin{Soutput}
[1] 15
\end{Soutput}
\end{Schunk}
\item Коэффицент AR:
\begin{Schunk}
\begin{Sinput}
> second <- 0
> mini <- 100500
> errCount <- 0
> for (m in 9:13)
+ {
+   a <- try(arima(data, order=c(m,0,first)))
+   if(class(a) == "try-error"){ 
+     errCount <- errCount + 1
+     next
+   }
+   cur = AIC(a)
+   if(cur < mini){
+     minInd = m
+     mini = cur
+     second <- minInd
+   }
+ }
\end{Sinput}
\end{Schunk}
AR-коэффицент получился равным 12:
\begin{Schunk}
\begin{Sinput}
> second
\end{Sinput}
\begin{Soutput}
[1] 11
\end{Soutput}
\end{Schunk}

\item Найдём оптимальную модель:

\begin{Schunk}
\begin{Sinput}
> mini <- 100500
> errCount <- 0
> third <- arima(data, order=c(second,0,first))
> for (m in 3:5)
+ {
+   a <- try(arima(data, order=c(12,m,15)))
+   if(class(a) == "try-error"){ 
+     errCount <- errCount + 1
+     next
+   }
+   cur = AIC(a)
+   if(cur < mini){
+     minInd = m
+     mini = cur
+     third <- a
+   }
+ }
\end{Sinput}
\end{Schunk}

\end{enumerate}

Снова проверим, соответствуют ли остатки модели <<белому шуму>>:

\begin{Schunk}
\begin{Sinput}
> tsdiag(third, 100)
\end{Sinput}
\end{Schunk}
\includegraphics{test-013}

Из графика видно, что остатки независисы (это видно по ACF). p-value гораздо больше уровня значимости, значит, остатки, с большой вероятностью, являются <<белым шумом>>. Значит, модель получилась хорошей.

\subsection{Обработка полученных данных и прогноз}

С помощью predict обработаем полученные данные. Полученные цифры прогноза:

\begin{Schunk}
\begin{Sinput}
> predict(third, n.ahead = 12, se.fit = TRUE)$pred
\end{Sinput}
\begin{Soutput}
           Jan       Feb       Mar       Apr       May       Jun       Jul
1979  8121.523  7718.303  8197.107  8580.422  9526.644  9882.218 10937.174
           Aug       Sep       Oct       Nov       Dec
1979 10303.080  9602.219  9419.893  9157.032  9809.976
\end{Soutput}
\end{Schunk}

На основании прогноза построим график с 1973 по 1979 годы:
\begin{Schunk}
\begin{Sinput}
> plot(data, xlim=c(1973,1981), ylim=c(min(data)-10000,max(data)))
> lines(predict(third, n.ahead=12, se.fit = TRUE)$pred, col="green")
\end{Sinput}
\end{Schunk}
\includegraphics{test-015}

Полученные цифры возможной ошибки:
\begin{Schunk}
\begin{Sinput}
> predict(third, n.ahead = 12, se.fit = TRUE)$se
\end{Sinput}
\begin{Soutput}
          Jan      Feb      Mar      Apr      May      Jun      Jul      Aug
1979 265.2952 326.6829 372.4956 436.9623 475.6924 517.4967 564.4794 585.3229
          Sep      Oct      Nov      Dec
1979 603.5406 645.0931 681.8046 736.2778
\end{Soutput}
\end{Schunk}

Добавим на график верхние и нижние границы прогноза:

\begin{Schunk}
\begin{Sinput}
> plot(data, xlim=c(1973,1980), ylim=c(min(data)-10000,max(data)))
> lines(predict(third, n.ahead=12, se.fit = TRUE)$pred, col="green")
> lines(predict(third, 
+               n.ahead=12, 
+               se.fit = TRUE)$se +
+       predict(third, n.ahead=12, se.fit = TRUE)$pred,
+       col="red")
> lines(-predict(third, n.ahead=12, se.fit = TRUE)$se +
+        predict(third, n.ahead=12, se.fit = TRUE)$pred,
+        col="blue")
\end{Sinput}
\end{Schunk}
\includegraphics{test-017}

\section{Вывод}

В ходе данной работы мы научились строить ARMA-модели и находить самые оптимальные из них, также, на основе полученной модели, мы научились строить прогнозы при помощи языка программировани R и инструмента R-studio из проекта R-project.
\end{document}
